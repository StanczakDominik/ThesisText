%%%%%%%%%%%%%%%%%%%%%%%%%%%%%%%%%%%%%%%%%%%%%%%%%%%%%%%%%%%%%%%%%%%%%%%%%%
% Template of thesis													 %
%																		 %
% Warsaw University of Technology										 %
% Faculty of physics													 %
% 																		 %
% Author of previous version: Michał Tarasiuk, tarasiuk.michal@gmail.com %
% Author: Patryk Bojarski, pbojarski94@gmail.com						 %
% Version: 2.0															 %
% Last modified: 26.12.2016												 %
%%%%%%%%%%%%%%%%%%%%%%%%%%%%%%%%%%%%%%%%%%%%%%%%%%%%%%%%%%%%%%%%%%%%%%%%%%

\documentclass[11pt,a4paper,twoside]{article}
\usepackage[includeheadfoot, left=3cm, right=2cm, top=2.5cm, bottom=2.5cm, headheight=19.3pt]{geometry}

\usepackage[polish]{babel}
\usepackage[T1]{fontenc}
\usepackage[utf8]{inputenc}
\usepackage[scaled]{helvet}
\renewcommand\familydefault{\sfdefault}

\let\lll=\relax
\usepackage{amsmath}
\usepackage{amsfonts}
\usepackage{amssymb}
\usepackage{bm}
\usepackage{booktabs}
\usepackage{url}
\usepackage{pdfpages}
\usepackage[bookmarks, pagebackref]{hyperref}
\usepackage[none]{hyphenat}
\usepackage{graphicx}
\usepackage{color}
\usepackage{array}
\usepackage{etoolbox, fancyhdr, xcolor}
\usepackage[hang, flushmargin]{footmisc}
\usepackage{setspace}
\usepackage{ragged2e}
\usepackage{MnSymbol}
\usepackage[nottoc]{tocbibind}
\usepackage{titlesec}
\usepackage[draft]{todonotes}
\urlstyle{rm}
\usepackage[ampersand]{easylist}
\usepackage{enumitem}
\setlist[itemize]{noitemsep, topsep=0pt, leftmargin=*, itemindent=-1cm}
\usepackage{epsfig}
\usepackage{float}
\usepackage[font={up, footnotesize}, labelfont=bf, singlelinecheck=off, format=hang]{caption}
\usepackage{caption}
\usepackage{subcaption}
\usepackage{listings}
\lstset{language=Python, rulecolor=\color{plum}, framerule=1.5pt} % you can change language of your code
\usepackage{colortbl}
\usepackage{tabu}
\usepackage{makecell}
\usepackage{boldline}
\setlength{\arrayrulewidth}{1pt}
\captionsetup[table]{name=Tabela}

\usepackage{lipsum} % for testing purposes

\newcommand{\concept}[1]{\emph{#1}}
\newcommand{\code}[1]{\texttt{#1}}
\newcommand{\english}[1]{\texttt{#1}}
\newcommand{\shape}[1]{(\texttt{#1})}
\newcommand{\pythonpic}[0]{\emph{PythonPIC}} % TODO space after stuff
% old color
\definecolor{red}{RGB}{194,0,11}
% new colors (since 2016)
\definecolor{plum}{RGB}{150,95,119}
\definecolor{grafit}{RGB}{60,60,60}

% pdf output setup
\hypersetup{
	unicode,
	pdftoolbar,
	pdfmenubar,
	pdffitwindow,
	pdfstartview = {FitH},
	pdftitle = {}, % <<<<<<<<<< title of pdf document
	pdfauthor = {}, % <<<<<<<<<< author of pdf document
	pdfnewwindow,
	colorlinks,
	linktoc = page,
	% use color plum or grafit
	linkcolor = plum,
	citecolor = plum,
	filecolor = plum,
	urlcolor = black
}

% itemize
\newcommand{\itemi}[1][plum]{\item[\color{#1} $\filledsquare$]}
\newcommand{\itemii}[1][plum]{\item[\color{#1} $\square$]}
\newcommand{\itemiii}[1][plum]{\item[\color{#1} $\bullet$]}

\ListProperties(Hide=100, Progressive=1cm, Style=\color{plum}, Style**=\color{black}, Style*=$\square$ ,Style2*=$\bullet$)

% vertical lines in header and footer
\newcommand{\headrulecolor}[1]{\patchcmd{\headrule}{\hrule}{\color{#1}\hrule}{}{}}
\newcommand{\footrulecolor}[1]{\patchcmd{\footrule}{\hrule}{\color{#1}\hrule}{}{}}

% information style
\newcommand{\infostyle}[1]
{
	\fancyhf{}
	\fancyhead[LO]{\Large{\textbf{#1}}}

	\renewcommand{\headrulewidth}{1.5pt}
	\headrulecolor{plum}
	\justify
	\pagestyle{fancy}
}

% footers and headers
\newcommand{\thesisstyle}
{
	\fancyhf{}
	\fancyhead[RO,LE]{\Large{\textbf{\rightmark}}}
	\fancyfoot[LE,RO]{\thepage}

	\renewcommand{\headrulewidth}{1.5pt}
	\renewcommand{\footrulewidth}{1.5pt}
	\headrulecolor{plum}
	\footrulecolor{plum}
	\justify
	\pagestyle{fancy}
}

% foot notes
\newcommand{\fancyfootnotetext}[2]
{
	\fancypagestyle{footnotes}
	{
    	\fancyfoot[LO,RE]{\parbox{15cm}{\footnotemark[#1]\footnotesize #2}}
	}
	\thispagestyle{footnotes}
}

% foot notes (2 footenotes)
\newcommand{\fancyfootnotetexts}[4]
{
	\fancypagestyle{footnotes}
	{
    	\fancyfoot[LO,RE]{\parbox{15cm}{\footnotemark[#1]\footnotesize #2 \\ \footnotemark[#3]\footnotesize #4}}
	}
	\thispagestyle{footnotes}
}

% foot notes (3 footenotes)
\newcommand{\fancyfootnotetextss}[6]
{
	\fancypagestyle{footnotes}
	{
    	\fancyfoot[LO,RE]{\parbox{15cm}{\footnotemark[#1]\footnotesize #2 \\ \footnotemark[#3]\footnotesize #4 \\ \footnotemark[#5]\footnotesize #6 \\}}
	}
	\thispagestyle{footnotes}
}

%spacing
\titlespacing\section{0pt}{15pt}{10pt} % spaces between section
\titlespacing\subsection{0pt}{15pt}{10pt} % spaces between subsection
\titlespacing\subsubsection{0pt}{15pt}{10pt} % spaces between subsubsection

\setstretch{1.15} % spacing between lines
\setlength{\parindent}{0.5cm} % indents
\setlength{\parskip}{0cm} % spacing between parahraphs

\frenchspacing
\sloppy

\author{Dominik Stańczak} % <<<<<<<<<< author of thesis
\title{Implementacja i analiza wydajności programu do symulacji Particle-in-Cell w języku Python} % <<<<<<<<<< title of thesis
\date{2017} % <<<<<<<<<< year
