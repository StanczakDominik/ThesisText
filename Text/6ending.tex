\section[Wnioski]{Wnioski}\label{sec:ending} % 3-5 stron
% 6. Zakończenia – będącego krótkim podsumowaniem realizacji pracy i
% rozwiązywanego zadania inżynierskiego. Zakończenie powinno zawierać:
% prezentację wniosków, odniesienie do poszczególnych rozdziałów pracy a także
% wskazanie na ew. rekomendowane kierunki dalszych prac nad podjętym zadaniem
% inżynierskim.
% 6. Zakończenie - 3-5 stron - Ponowna krótka prezentacja wyników podsumowująca
%   pracę z licznymi odniesieniami do rozdziałów pracy

W niniejszej pracy opisano proces tworzenia kodu symulacyjnego implementującego
algorytm particle-in-cell, którego sens i kontekst w szerszej fizyce plazmy tłumaczy rozdział~\ref{sec:intro}.

W rozdziale~\ref{sec:theory} wyjaśniono zasadność modelowania plazmy symulacjami komputerowymi oraz zasadność przyjętego przybliżenia
% TODO CZY JA WYJAŚNIŁEM ZASADNOŚĆ PRZYJĘTEGO PRZYBLIŻENIA NA PODSTAWIE BIRDSALLA INTRO?
modelowania plazmy dyskretnymi cząstkami. Poszczególne elementy realizacji algorytmu particle-in-cell w bieżącym kodzie wyjaśniono w rozdziale~\ref{sec:implementation}.

W rozdziale~\ref{sec:code} opisano ekosystem Pythona, jego potencjał dla nauki i symulacji numerycznej, a także strukturę i hierarchię samego kodu, który jest przedmiotem bieżącej pracy.

Z dobrą zgodnością z zakładanym modelem i istniejącymi programami symulacyjnymi udało się wykonać symulację zachowania plazmy pobudzonej wysokoenergetycznym impulsem laserowym, jak pokazuje rozdział~\ref{sec:verification}.

Przeprowadzone w rozdziale~\ref{sec:profiling} benchmarki pokazują, że kod będący przedmiotem bieżącej pracy jest o około rząd wielkości wolniejszy niż jego najbliższy możliwy odpowiednik
napisany w C++, jeżeli kompilacja przeprowadzana jest z użyciem najprostszej flagi optymalizacyjnej. Użycie flagi \code{-O2} bądź wyższych skutkuje uzyskaniem dalszych postępów w szybkości działania programu.
Oczywiście, znaczącym czynnikiem może być wspomniane w rozdziale~\ref{sec:profiling} wykorzystanie bardziej niskopoziomowego paradygmatu iteracji po indeksach cząstek w kodzie w C++, spowodowane
brakami w bibliotece Eigen.

Istnieje możliwość, że do poprawy wydajności programu konieczna jest reimplementacja najbardziej złożonych częsci algorytmu w Pythonie (relatywistycznego integratora Borysa oraz depozycji prądu) w C++
bądź w czystym, prostym Pythonie z kompilacją JIT poprzez bibliotekę Numba (faktycznie, takie rozwiązanie zdaje się skutkować dobrą wydajnością w kodzie~\cite{fbpic}).

Innym rozwiązaniem byłoby ręczne zrównoleglenie obliczeń w programie przy użyciu mocy obliczeniowej kart graficznych. Potencjalnymi kandydatami do tego celu są popularne biblioteki \code{PyCUDA} bądź \code{Numba.cuda}.

Alternatywnym sposobem zrównoleglenia kodu jest wykorzystanie obliczeń na CPU poprzez paradygmaty OpenMP (na wielu wątkach obliczeniowych mających dostęp do tej samej pamięci)
bądź MPI (na dużej liczbie niezależnych procesorów mających przydzielone własne obszary pamięci, komunikujące się za pomocą oddzielnego interfejsu). Niestety, obsługa wielu intensywnych obliczeniowo wątków w Pythonie
jest znacząco utrudniona przez tzw. GIL (\english{Global Interpreter Lock}), czego efektem w praktyce byłaby konieczność implementacji tych fragmentów w C++ bądź wykorzystanie Cythona (alternatywnej wersji Pythona, automatycznie tłumaczonej na C).

Program w obecnej postaci działa w jednym wymiarze, w reżimie relatywistycznym i jako taki może służyć do prostego modelowania oddziaływania plazmy z wiązkami laserowymi
bądź do modelowania niestabilności w symulacjach elektrostatycznych.

Oczywistym kierunkiem rozwoju dla kodu jest dodanie obsługi ruchu cząstek w wielu wymiarach. Wymagałoby to niestety znacznej zmiany modelu rozwiązywania równań ewolucji czasowej pola elektromagnetycznego.
Metodę rozwiązywania warunków początkowych dla tego pola można zaadaptować z istniejącego w kodzie rozwiązania dla pola elektrostatycznego. Jednakże, gdy układ straci symetrię typu
$\frac{\partial}{\partial y} = \frac{\partial}{\partial z} = 0$ założoną w modelowaniu 1D, będzie konieczne również rozwiązywanie
prawa Gaussa dla magnetyzmu~\ref{eqn:maxwell-B-div}, aby znaleźć początkowy rozkład pola magnetycznego. W tej chwili prawo to sprowadza się do $\frac{\partial B_x}{\partial x} = 0$, co
w jednym wymiarze jest spełnione poprzez istniejący w symulacji warunek $B_x = 0$.

Symulacja nie uwzględnia również kolizji. Możliwe jest dodanie ich na przykład poprzez sprzężenie algorytmu z implementacją DSMC (\english{Direct Simulation Monte Carlo})~\cite{particleincell-dmsc}.

W bieżącej symulacji użyto prostego modelu pojedynczo zjonizowanych atomów wodoru bez możliwości syntezy w neutralne cząstki wodoru. Istnieje możliwość wybrania atomów innych pierwiastków,
lecz w chwili obecnej kod zupełnie nie uwzględnia możliwości dalszej jonizacji tych cząstek.
