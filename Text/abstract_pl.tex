\section{Streszczenie}

Język programowania Python zdobywa coraz większą popularność w fizyce obliczeniowej.
Jedną z najbardziej intensywnych obliczeniowo dziedzin fizyki jest symulacyjna fizyka
plazmy, czyli układów wielu cząstek obdarzonych ładunkiem elektrycznych.
Niniejsza praca podejmuje próbę przetestowania możliwości Pythona wykorzystującego 
obliczenia macierzowe w tych zastosowaniach.

W tym celu
utworzono kod symulacyjny Particle-in-Cell mający modelować interakcję relatywistycznej plazmy wodorowej oraz
impulsu laserowego. Kod zoptymalizowano poprzez wykorzystanie ogólnodostępnych bibliotek 
wywołujących niskopoziomowe procedury numeryczne do osiągnięcia wysokiej wydajności obliczeniowej. Przeprowadzono 
analizę wydajności i skalowania programu.
