\section{Streszczenie}

Język programowania Python zdobywa coraz większą popularność w fizyce obliczeniowej.
Jedną z najbardziej intensywnych obliczeniowo dziedzin fizyki jest symulacyjna fizyka
plazmy, czyli układów wielu cząstek obdarzonych ładunkiem elektrycznych.
Przykładowym rodzajem algorytmu symulacyjnego w tej dziedzinie jest metoda particle-in-cell
(``cząstka w komórce'').
Niniejsza praca podejmuje próbę przetestowania możliwości Pythona wykorzystującego 
obliczenia macierzowe w tych zastosowaniach.

W tym celu
utworzono jednowymiarowy kod symulacyjny Particle-in-Cell mający modelować interakcję relatywistycznej plazmy wodorowej oraz
impulsu laserowego. Kod zoptymalizowano poprzez wykorzystanie ogólnodostępnych bibliotek 
wywołujących niskopoziomowe procedury numeryczne do osiągnięcia wysokiej wydajności obliczeniowej. Na jego podstawie
napisano również analogiczny program w języku C++ dla porównania wydajności.

Uzyskany program zwraca dobre jakościowo wyniki, zbliżone do rezultatów z istniejących symulacji i spodziewanych
wyników teoretycznych dla symulacji elektrostatycznych oraz dla symulacji interakcji tarczy wodorowej z wiązką laserową
na bazie symulacji eksperymentów z Instytutu Fizyki Plazmy i Laserowej Mikrosyntezy.

Przeprowadzono analizę wydajności i skalowania programu w obu wersjach z liczbą cząstek.  
Jak pokazują benchmarki, implementacja w Pythonie mimo prób optymalizacji w wybranym paradygmacie wysokopoziomowych obliczeń macierzowych
uzyskuje wyniki zadowalające, lecz gorsze niż reimplementacja w C++ przy
kompilacji z optymalizacją o rząd wielkości w kwestii szybkości obliczeń.
