\section{Abstract} % 150-250słów, polski + angielski

The Python programming language has recently been growing in popularity in computational physics.
One of the most computationally intensive branches of physics is computational plasma physics,
which deals with systems of many charged particles.
An example of a simulation algorithm in this area is the particle-in-cell method.
This work attempts to test the capabilities of Python in these applications
using a matrix calculation approach. 

To this end, a one-dimensional Python particle-in-cell plasma simulation code is developed to
model the interaction between a hydrogen plasma target
and a laser impulse. The code is then optimized via using the high level Python programming
language to call low level numerical procedures from widely available
libraries. Based on this code, another program in C++ is developed to compare performance and
scaling between implementations.

The developed program does well on the qualitative side, returning results comparable to existing
simulations and theoretical results in the field for electrostatic and electromagnetic cases.

The performance and scaling analysis carried out in the latter part of this work shows
that despite attempts at optimizing the chosen high-level matrix calculation paradigm,
the developed Python code runs decently, yet more slowly than the C++ reimplementation (assuming compilation
with optimization flags) by an order of magnitude in speed.
