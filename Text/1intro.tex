\section[Wstęp]{Wstęp} %2-3 strony wprowadzenie w temat, motywacja, teza (cel)
Algorytmy Particle-in-Cell (cząstka w komórce) to jedne z najbardziej zbliżonych do fundamentalnej fizyki
metod symulacji materii w stanie plazmy. Zastosowany w nich lagranżowski opis cząsteczek pozwala na dokładne
odwzorowanie dynamiki ruchu elektronów i jonów. Jednocześnie, ewolucja pola elektromagnetycznego na Eulerowskiej
siatce dokonywana zamiast bezpośredniego obliczania oddziaływań międzycząsteczkowych pozwala na znaczące
przyspieszenie etapu obliczenia oddziaływań międzycząsteczkowych. W większości symulacji cząsteczkowych właśnie
ten etap jest najbardziej krytyczny dla wydajności progamu.

W ostatnich czasach symulacje Particle-in-Cell zostały wykorzystane między innymi do
\begin{enumerate}
\item symulacji przewidywanej turbulencji plazmy w reaktorze termojądrowym ITER \cite{pic-hammett}
\item modelowania rekonekcji linii magnetycznych w polu gwiazdy \cite{pic-reconnection}
\item projektowania silników jonowych (Halla) \cite{pic-hallengine}
\item badania interakcji laserów z plazmą w kontekście tworzenia niewielkich,
    wysokowydajnych akceleratorów cząstek \cite{pic-laserplasma}
\end{enumerate}

    Należy zauważyć, że w świetle rosnącej dostępności silnie równoległej mocy obliczeniowej w postaci kart graficznych
    możliwości algorytmów Particle-in-Cell będą rosły współmiernie, co może pozwolić na rozszerzenie zakresu ich zastosowań.
    Przykładem takiego projektu jest PIConGPU \cite{picongpu}

    Inżynieria oprogramowania zorientowanego na wykorzystanie możliwości kart graficznych,
    jak również w ogólności nowoczesnych symulacji wykorzystujących dobrodziejstwa nowych technologii
    jest jednak utrudniona poprzez niskopoziomowość istniejących języków klasycznie
    kojarzonych z symulacją numeryczną (C, FORTRAN) oraz istniejących technologii zrównoleglania
    algorytmów (MPI, CUDA, OpenCL).

    Należy też zauważyć, że programy takie często są
    trudne, jeżeli nie niemożliwe do weryfikacji działania, ponownego wykorzystania
    i modyfikacji przez osoby niezwiązane z oryginalnym autorem z powodów takich jak
    \begin{itemize}
        \item brak dostępności kodu źródłowego
        \item niedostateczna dokumentacja
        \item brak jasno postawionych testów pokazujących, kiedy algorytm działa zgodnie z zamiarami twórców
        \item zależność działania kodu od wersji zastosowanych bibliotek, sprzętu i kompilatorów
    \end{itemize}

    To sprawia, że pisanie złożonych programów symulacyjnych, zwłaszcza przez osoby
    zajmujące się głównie pracą badawczą (na przykład fizyką), bez dogłębnego, formalnego przeszkolenia
    w programowaniu, jest znacznie utrudnione.

    Niniejsza praca ma na celu utworzenie kodu symulacyjnego wykorzystującego metodę Particle-in-Cell
    do symulacji oddziaływania wiązki laserowej z tarczą wodorową w popularnym języku
    wysokopoziomowym Python, przy użyciu najlepszych praktyk tworzenia reprodukowalnego, otwartego oprogramowania
    i zoptymalizowanie go w celu osiągania maksymalnej wydajności i sprawności obliczeniowej.

    Może to też oczywiście pozwolić na dalsze
    zastosowanie kodu w celach badawczych i jego dalszy rozwój, potencjalnie z użyciem kart graficznych.

    Ostatecznie, jest to również test wydajnościowy możliwości Pythona w intensywnych symulacjach
    \todo[inline]{Zamiast ostatecznie chciałbym dać coś typu last but not least}
    numerycznych, które do tej pory były często domeną komwencjonalnie
    zarezerwowaną dla języków niskopoziomowych.
    \todo[inline]{Tu przydałoby się przynajmniej do dwóch pełnych stron pod objętość}
