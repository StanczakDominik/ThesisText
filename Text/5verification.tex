\section[Weryfikacja]{Część weryfikacyjna}\label{sec:verification} % 30-40% - opis wyników, analiza, weryfikacja i porównanie do danych literaturowych
    Niniejsza analiza przeprowadzona została na ``finalnej'' w chwili pisania
    niniejszej pracy wersji programu.  W repozytorium Git na GitHubie jest to
    commit \code{480154c9c8d21350e8761e3320ede5f63bbb9ef5}.

    \subsection{Przypadki testowe}

    Kod przetestowano w dwojaki sposób. Pierwszym z nich są testy jednostkowe.
    Automatyczne testy jednostkowe uruchamiane po każdej wymiernej zmianie kodu
    pozwalają kontrolować działanie programu znacznie ułatwiają zapobieganie
    błędom.

    Poszczególne algorytmy podlegały testom przy użyciu ogólnodostępnego
    pakietu \code{pytest}~\ref{sec:pytest} i w większości były
    uruchamiane na platformie TravisCI\@.

    \subsubsection{Testy algorytmiczne}
    Testy algorytmiczne polegały na przeprowadzeniu fragmentu symulacji --- w
    przypadku testów algorytmów było to na przykład wygenerowanie pojedynczej
    cząstki o jednostkowej prędkości oraz zdepozytowanie jej gęstości prądu na
    siatkę, co pozwala porównać otrzymany wynik z przewidywanym analitycznie
    dla danego rozmiaru siatki i położenia cząstki.
    \begin{enumerate}
        \itemi{} \english{Gather} --- \code{test\_current\_deposition}, \code{test\_charge\_deposition}
            \begin{enumerate}
                \itemii{} Depozycja prądu lub ładunku z pojedynczej cząstki na niewielką siatkę i porównanie z wynikiem analitycznym\cite{Jablonski-notes}
                \itemii{} Depozycja prądu lub ładunku z dwóch pojedynczych cząstek na niewielką
                    siatkę i porównanie z sumą prądów lub ładunków dla obu pojedynczyczh
                    cząstek
                \itemii{} Depozycja prądu z dużej ilości równomiernie rozłożonych
                    cząstek --- weryfikacja średniej wartości prądu
                \itemii{} Depozycja prądu lub ładunku z cząstek blisko granic i porównanie z zamierzonymi warunkami brzegowymi
                \itemii{} Depozycja prądu dla cząstki poruszającej się powyżej prędkości światła --- procedura depozycji zwraca błąd
            \end{enumerate}

        \itemi{} \english{Solve} --- \code{test\_FieldSolver}, \code{test\_LongitudinalSolver}
            \begin{enumerate}
                \itemii{} Test \english{solvera} spektralnego poprzez
                    rozwiązanie równania Poissona dla sinusoidalnej, deltowej i rosnącej liniowo
                    dystrybucji ładunku. Weryfikacja dokładności rozwiązania i energii pola
                \itemii{} Test \english{solvera} rotacyjnego poprzez porównanie z rezultatami analitycznymi dla pojedynczych iteracji
            \end{enumerate}

        \itemi{} \english{Scatter} --- \code{test\_field\_interpolation} --- porównanie interpolowanego pola z przypadkiem analitycznym dla kilku prostych rozkładów pola

        \itemi{} \english{Push} --- \code{test\_pusher}
            \begin{enumerate}
                \itemii{} Ruch w jednorodnym polu elektrycznym wzdłuż osi układu,
                    w wersji nierelatywistycznej i relatywistycznej. Prędkość cząstki rośnie
                    jednostajnie w reżimie nierelatywistycznym.
                \itemii{} Relatywistyczny i nierelatywistyczny ruch kołowy z
                    zachowaniem energii w jednorodnym polu magnetycznym w kierunku osi $\hat{z}$. 
                \itemii{} Relatywistyczny ruch w potencjale oscylatora harmonicznego. Ruch cząstki jest porównany
                    z trajektorią obliczoną analitycznie w przypadku relatywistycznym.
                \itemii{} Zachowanie energii ruchu cząstki bez pola elektrycznego przy wysoce relatywistycznej prędkości, w kierunku ruchu oraz z prędkościami
                    w trzech wymiarach.
                \itemii{} Test dryfu $\vec{E} \times \vec{B}$ --- cząstka w skrzyżowanych polach elektrycznym i magnetycznym uzyskuje stabilną prędkość $E/B$.
                \itemii{} Test warunków brzegowych dla symulacji okresowych --- zachowanie liczby cząstek przechodzących przez granicę obszaru symulacji.
                \itemii{} Test warunków brzegowych dla symulacji okresowych --- spadek liczby cząstek przechodzących przez granicę obszaru symulacji do zera.
            \end{enumerate}
    \end{enumerate}

    \subsection{Testy symulacyjne --- przypadki elektrostatyczne}
    Testy symulacyjne polegały na uruchomieniu niewielkiej symulacji testowej z
    różnymi warunkami brzegowymi i ilościowym, automatycznym bądź jakościowym, wizualnym zweryfikowaniu
    dynamiki zjawisk w niej zachodzących.

    Zastosowano kod do symulacji kilku znanych problemów w fizyce plazmy:
    \subsubsection{oscylacje zimnej plazmy}
    Jest to efektywnie elektrostatyczna fala stojąca. Symulacja zaczyna z ujemnymi cząstkami
o zerowej prędkości początkowej, rozłożonymi w okresowym pudełku symulacyjnym
równomiernie z nałożonym na nie sinusoidalnym zaburzeniem:

\begin{align}
x &= x_0 + x_1\\
x_0 &= L * n / N\\
x_1 &= A  \sin(k x_0) = A \sin(2 \pi n x_0 / L)
\end{align}

Określenie ``zimna plazma'' bierze się z nietermalnego, deltowego
rozkładu prędkości cząstek --- jest to faktycznie strumień cząstek o stałej
(w tym szczególnym przypadku zerowej) prędkości.

Sumaryczny ładunek w obszarze symulacji jest ustawiony na zero w pierwszym
kroku algorytmu rozwiązywania pola elektrycznego poprzez wyzerowanie zerowej
składowej fourierowskiej gęstości ładunku, co jest jednoznaczne z przyjęciem
nieskończenie masywnych i nieruchomych jonów dodatnich dokładnie
neutralizujących gęstość ładunku elektronów.

Sytuacja ta
pozwala na obserwację oscylacji cząstek wokół ich stabilnych położeń
równowagi. W przestrzeni fazowej $x, V_x$ cząstki zataczają efektywnie
elipsy, co pozwala wnioskować że ruch ten jest z dobrym przybliżeniem harmoniczny.
Oczywiście, nie jest to do końca oscylacja harmoniczna z powodu odchyleń pola interpolowanego
z Eulerowskiej siatki od generowanego faktycznym potencjałem $ \sim x^2 $.

%Jest to spełnione jedynie dla niewielkich odchyleń; dla $A \to
%dx$ obserwuje się nieliniowy reżim oscylacji, 

Symulacja ta jest wykorzystywana do weryfikacji podstawowych warunków, jakie powinna spełniać
symulacja elektrostatyczna --- na przykład długofalowe zachowanie energii, liczby cząstek (w układzie okresowym cząstki nie powinny
znikać).

\begin{figure}[h!]
  \includegraphics[width=\textwidth]{Images/COSCALING}
  \caption{Oscylacje zimnej plazmy w reżimie liniowym. Cząstki wykonują prawie harmoniczne oscylacje wokół swoich położeń równowagi, zaś energia
   w układzie jest wymieniana między energią kinetyczną cząstek a energią pola
elektrostatycznego. Jako wyjście programu, wykresy z \pythonpic{} w bieżącym rozdziale zamieszczane są w języku angielskim.\label{fig:coldplasma-linear}}

\end{figure}

Liniowy reżim obserwacji jest zaznaczony na rysunku~\ref{fig:coldplasma-linear}.%, zaś nieliniowy~\ref{fig:coldplasma-nonlinear}.

    \subsubsection{niestabilność dwóch strumieni}
W tym przypadku symulacja również zawiera zimną plazmę, lecz tym razem są to dwa strumienie ujemnych cząstek
o stałych, przeciwnych sobie prędkościach $v_0$ oraz $-v_0$.

    Dla niewielkich prędkości początkowych strumieni obserwuje się
    liniowy reżim oscylacji cząstek --- oba strumienie pozostają stabilne. Obserwuje się niewielkie oscylacje oraz
grupowanie się cząstek w rejony koherentnej większej gęstości wewnątrz strumienia (opisany przez Birdsalla i Langdona \emph{bunching}).

    Dla dużych prędkości obserwuje się nieliniowe
    zachowanie cząstek w przestrzeni fazowej. Oscylacje prędkości cząstek przybierają rząd wielkości porównywalny
    z początkową różnicą prędkości strumieni.
 Strumienie zaczynają się mieszać ze sobą nawzajem, zaś cały układ się termalizuje. Energia kinetyczna
 uporządkowanego ruchu strumieni zamienia się w energię potencjalną pola równowagowego 
oraz termalną energię kinetyczną, co sprawia, że średnia prędkość obu strumieni ulega zmniejszeniu.\cite{birdsall}

To oraz stabilność symulacji o warunkach początkowych w obu reżimach jest
obiektem automatycznych testów sprawdzających poprawność symulacji.

\begin{figure}[h!]
  \includegraphics[width=\textwidth]{Images/TS_STABLE}
  \caption{Stabilny reżim dla układu dwóch strumieni. Widoczne są jedynie cykliczny ruch jednostajny w obszarze symulacji i niewielkie zmiany prędkości cząstki.\label{fig:twostream-stable}}
\end{figure}

\begin{figure}[h!]
  \includegraphics[width=\textwidth]{Images/TS_UNSTABLE}
  \caption{Niestabilny reżim dla układu dwóch strumieni. Trajektoria cząstki staje się chaotyczna.\label{fig:twostream-unstable}}
\end{figure}

\subsection{Testowe symulacje elektromagnetyczne --- propagacja fali}
W tym przypadku symulacja nie zawiera plazmy, a badana jest jedynie propagacja fali elektromagnetycznej w obszarze
symulacji dla różnych charakterystyk czasowych. 

Testy jednostkowe obejmują zachowanie energii weryfikowane poprzez zgodność
energii pola elektromagnetycznego w symulacji ze strumieniem Poyntinga generowanym
przez warunek brzegowy. % TODO POYNTING FLUX

% TODO RYSUNEK

\subsection{Symulacja elektromagnetyczna --- oddziaływanie wiązki laserowej z tarczą wodorową}

Jest to sytuacja, która od początku rozwoju programu miała być głównym celem istnienia kodu. Szerzej została opisana w~\ref{seq:lasershield}.

Jako warunki początkowe przyjęto plazmę rozbitą na dwie części ---
\emph{preplazmę} o narastającej funkcji rozkładu gęstości oraz plazmę właściwą
o stałej gęstości. Funkcja gęstości jest generowana automatycznie poprzez
metodę opisaną w~\cite{birdsall} i jest normalizowana do danego poziomu
maksymalnej gęstości w obszarze plazmy właściwej przy zadanej liczbie
makrocząstek.

Początkowe prędkości cząstek przyjęto jako zerowe, zaś początkowe położenia
elektronów są identyczne z położeniami początkowymi protonów. Jest to prosta
w inicjalizacji, inherentnie równowagowa sytuacja (gwarantująca nie tylko
kwaziobojętność elektryczną plazmy, ale również całkowitą obojętność
początkowego układu) do czasu przybycia lasera. Należy również zauważyć, że
przy zakładanych energiach lasera jakakolwiek termalna dynamika układu przed
interakcją zostanie szybko wymazana. Jest to więc akceptowalne przybliżenie.
%IDEA wylosowano z relatywistycznego rozkładu Maxwella w kierunkach y, z

Za intensywność lasera przyjęto wielkości $10^{21}, 10^{22}, 10^{23} W/m^2$,
zaś za jego długość fali 1.064 $\mu$m (jest to laser Nd:YAG). Przeprowadzono
badania w polaryzacjach liniowej (pole elektryczne w kierunku osi $\hat{y}$ ---
nie zaobserwowano znaczących różnic w symulacji, w której pole elektryczne
skierowano w kierunku osi $\hat{z}$)
oraz kołowej wiązki laserowej.

Długość obszaru symulacji to około $10.6 \mu$m, podzielone na 1378 komórek siatki.

Dla mocy lasera do $10^{22} J/m^2$ przy polaryzacji liniowej % kołowej?
laser zostaje w większości wytłumiony i odbity przez plazmę. Obserwuje się generację niewielkich oscylacji wysokiej częstotliwości emitowanych w kierunku $+\hat{x}$.
Widać to zwłaszcza w przypadku polaryzacji kołowej.

Dla wszystkich przypadków od $10^{22} J/m^2$ obserwuje się spłaszczenie preplazmy w kierunku plazmy głównej.

% comment

% TODO rysunek ilustrujący zjawisko

Dla mocy lasera $10^{23} J/m^2$ przy polaryzacji liniowej % kołowej?
obserwuje się znaczące przejście wiązki laserowej przez warstwę plazmy oraz silne zniszczenie początkowego rozkładu przestrzennego plazmy, z przejściem do termalizacji.
Gęstości obu rodzajów cząstek uzyskują rozkład normalny.

Wizualizacje symulacji tarczy wodorowej przedstawiono na rysunkach~\ref{fig:laser-21-Ey}--\ref{fig:laser-23-Ey-snapshot}.


% Dla polaryzacji kołowej zaobserwowano występowanie siły ponderomotorycznej~\cite{Jablonski-notes}


%\begin{equation}
    %F_p = - \frac{e^2}{4 m \omega^2} \nabla (\vec{E}^2)
    %\label{eqn:ponderomotive}
%\end{equation}
% https://en.wikipedia.org/wiki/Ponderomotive_force


%Dynamika układu

\begin{figure}[h!]
  \includegraphics[width=\textwidth]{Images/75000_1378_run_21_Ey}
  \caption{Intensywność wiązki laserowej $10^{21} J/m^2$, polaryzacja liniowa.\label{fig:laser-21-Ey}}
\end{figure}

\begin{figure}[h!]
  \includegraphics[width=\textwidth]{Images/75000_1378_run_21_Ey_004700}
  \caption{Intensywność wiązki laserowej $10^{21} J/m^2$, polaryzacja liniowa. Rzut na iterację 4700/7755.\label{fig:laser-21-Ey-snapshot}}
\end{figure}

\begin{figure}[h!]
  \includegraphics[width=\textwidth]{Images/75000_1378_run_21_Circular}
  \caption{Intensywność wiązki laserowej $10^{21} J/m^2$, polaryzacja kołowa.\label{fig:laser-21-Circular}}
\end{figure}

\begin{figure}[h!]
  \includegraphics[width=\textwidth]{Images/75000_1378_run_21_Circular_004700}
  \caption{Intensywność wiązki laserowej $10^{21} J/m^2$, polaryzacja kołowa. Rzut na iterację 4700/7755.\label{fig:laser-21-Circular-snapshot}}
\end{figure}

\begin{figure}[h!]
  \includegraphics[width=\textwidth]{Images/75000_1378_run_23_Ey}
  \caption{Intensywność wiązki laserowej $10^{23} J/m^2$, polaryzacja liniowa.\label{fig:laser-23-Ey}}
\end{figure}

\begin{figure}[h!]
  \includegraphics[width=\textwidth]{Images/75000_1378_run_23_Ey_004700}
  \caption{Intensywność wiązki laserowej $10^{23} J/m^2$, polaryzacja liniowa. Rzut na iterację 4700/7755.\label{fig:laser-23-Ey-snapshot}}
\end{figure}


\begin{figure}[h!]
  \includegraphics[width=\textwidth]{Images/75000_1378_run_23_Circular}
  \caption{Intensywność wiązki laserowej $10^{23} J/m^2$, polaryzacja kołowa.\label{fig:laser-23-Circular}}
\end{figure}

\begin{figure}[h!]
  \includegraphics[width=\textwidth]{Images/75000_1378_run_23_Circular_004700}
  \caption{Intensywność wiązki laserowej $10^{23} J/m^2$, polaryzacja kołowa. Rzut na iterację 4700/7755.\label{fig:laser-23-Circular-snapshot}}
\end{figure}


\subsection{Parametry uruchomienia}

Profilowanie przeprowadzono na komputerze o parametrach:
\begin{itemize}
\itemi{} Intel Core 4 Quad 6600 3.0GHz
\itemi{} 4GB RAM
\itemi{} SSD firmy Crucial % TODO CHECK 
\itemi{} Arch Linux
\end{itemize}

Wykorzystując następujące wersje oprogramowania i bibliotek obliczeniowych:
\begin{itemize}
\itemi{} Python 3.6.1
\itemi{} Numpy 1.13.1
\itemi{} Numba 0.34.0
\itemi{} h5py 2.7.0
\itemi{} SciPy 0.19.1
\end{itemize}

