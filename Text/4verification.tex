\section[Weryfikacja]{Część weryfikacyjna} % 30-40% - opis wyników, analiza, weryfikacja i porównanie do danych literaturowych
    Niniejsza analiza przeprowadzona została na ``finalnej'' w chwili pisania
    niniejszej pracy wersji programu.  W repozytorium Git na Githubie jest to
    commit ``placeholder'' \todo[inline]{uzupełnić commita} identyfikowany
    również jako wersja 1.0.

    \subsection{Przypadki testowe}

    Kod przetestowano w dwojaki sposób. Pierwszym z nich są testy jednostkowe.
    Automatyczne testy jednostkowe uruchamiane po każdej wymiernej zmianie kodu
    pozwalają kontrolować działanie programu znacznie ułatwiają zapobieganie
    błędom.

    Poszczególne algorytmy podlegały testom przy użyciu ogólnodostępnego
    pakietu \code{pytest} \todo[inline]{pytest reference.} i w większości były
    uruchamiane na platformie TravisCI.

    \subsubsection{Testy algorytmiczne}
    Testy algorytmiczne polegały na przeprowadzeniu fragmentu symulacji - w
    przypadku testów algorytmów było to na przykład wygenerowanie pojedynczej
    cząstki o jednostkowej prędkości oraz zdepozytowanie jej gęstości prądu na
    siatkę, co pozwala porównać otrzymany wynik z przewidywanym analitycznie
    dla danego rozmiaru siatki i położenia cząstki.
    \todo[inline]{sprawdzić listę testów}
    \begin{enumerate}
        \itemi Gather
            \begin{enumerate}
                \item Depozycja prądu z pojedynczej cząstki na niewielką siatkę
                \item Depozycja prądu z dwóch pojedynczych cząstek na niewielką
                    siatkę i porównanie z sumą prądów dla obu pojedynczyczh
                    cząstek
                \item Depozycja prądu z dużej ilości równomiernie rozłożonych
                    cząstek
            \end{enumerate}

        \itemi Solve
            \begin{enumerate}
                \item Symulacja fali sinusoidalnej, obwiedni impulsu i złożenia
                    tych dwóch propagujących się w próżni
            \end{enumerate}

        \itemi Scatter
            \begin{enumerate}
                \item \ldots \todo[inline]{write these}
            \end{enumerate}

        \itemi Push
            \begin{enumerate}
                \itemii Ruch w jednorodnym polu elektrycznym wzdłuż osi układu
                \itemii Ruch w jednorodnym polu magnetycznym z polem
                    magnetycznym
            \end{enumerate}
    \end{enumerate}

    \subsubsection{Testy symulacyjne}
    Testy symulacyjne polegały na uruchomieniu niewielkiej symulacji testowej z
    różnymi warunkami brzegowymi i ilościowym, automatycznym zweryfikowaniu
    dynamiki zjawisk w niej zachodzących.

    Zastosowano kod do symulacji kilku znanych problemów w fizyce plazmy:
    \subsubsection{oscylacje zimnej plazmy}
    Jest to efektywnie fala stojąca. Jednorodne rozmieszczenie cząstek z zerową
    prędkością początkową (stąd określenie "zimna plazma" jako nietermalna)
    \todo[inline]{czy ja ruszam prędkości czy położenia i czy to nie powinno
    zmienić fazy} jednego typu na okresowej siatce z jednoczesnym wysunięciem
    ich z położeń równowagi o $\Delta x = A \sin(kx)$, gdzie $k = n 2 \pi / L$,
    pozwala na obserwację oscylacji cząstek wokół ich stabilnych położeń
    równowagi. W przestrzeni fazowej $x, V_x$ cząstki zataczają efektywnie
    elipsy, co pozwala wnioskować że ruch ten jest harmoniczny.

    Jest to, oczywiście, spełnione jedynie dla niewielkich odchyleń; dla $A \to
    dx$ \todo[inline]{dx} obserwuje się nieliniowy reżim \todo[inline]{i co}

    Jest to też łoże testowe \todo[inline]{sformułowanie} dla innych
    przypadków, takich jak efekt Kaiser-Wilhelm
    \todo[inline]{sformułowanie z BL}
    oraz \todo[inline]{czegoś jeszcze.}
    \subsubsection{niestabilność dwóch strumieni}
    \todo[inline]{przeformułować}
    Różnice między tym a poprzednim przypadkiem to obecność dwóch jednorodnie
    rozłożonych strumieni cząstek z przeciwnie skierowanymi prędkościami wzdłuż
    osi układu.

    Dla niewielkich prędkości \todo[inline]{sparametryzować} obserwuje się
    liniowy reżim \todo[inline]{bunchingu}

    Dla dużych prędkości \todo[inline]{sprawdzić} obserwuje się nieliniowe
    zachowanie cząstek, które zaczynają się mieszać ze sobą, zaś cały układ się
    termalizuje.
     \todo[inline]{opisać dalej}
    \subsection{Symulacja oddziaływania lasera z tarczą wodorową}

    Jako warunki początkowe przyjęto plazmę z liniowo narastającą funkcją
    rozkładu gęstości (jest to tak zwany obszar prejonizacji)
    \todo[inline]{preplazmy?}

    Gęstość rozkładu plazmy przyjęto jako

    Początkowe prędkości cząstek przyjęto jako zerowe.
    \todo[inline]{wylosowano z relatywistycznego rozkładu Maxwella w kierunkach y, z}

    Za moc lasera przyjęto $10^{23} W/m^2$, \todo[inline]{ASK: czy to nie jest
    za dużo?} zaś za jego długość fali 1.064 $\mu$m (jest to laser Nd:YAG)

    Długość obszaru symulacji to \todo[inline]{wstawić}.

    Prędkość światła $c$, stałą dielektryczną $\varepsilon_0$, ładunek
    elementarny $e$, masy protonu i elektronu $m_p$, $m_e$ przyjęto według
    tablic, jak obrazuje następująca tabela:

    \todo[inline]{zrobić tabelkę na stałe}

    \subsection{Benchmarki - szybkość, zasobożerność} \todo[inline]{fix}
     Do przeprowadzenia testów wydajności kodu użyto \todo[inline]{cProfile}
    \subsection{Problemy napotkane w trakcie pisania kodu}
