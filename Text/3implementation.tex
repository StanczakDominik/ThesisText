\section[Implementacja]{Implementacja}% 20-30% - opis przyjętych rozwiązań i uzasadnienie ich wyboru
    \subsection{Zastosowane algorytmy}
    \subsubsection{Leapfrog oraz Borys}
    Każda symulacja cząstek wymaga zastosowania integratora równań ruchu.
    Tradycyjnym przykładem takiego integratora jest integrator Rungego-Kutty
    czwartego rzędu, znajdujący zastosowanie w wielorakich symulacjach.
    \todo[inline]{reference}

    Niestety, w bieżącym kodzie nie można go zastosować ze względu na jego
    niesymplektyczność: mimo ogromnej dokładności jest on niestabilny pod
    względem energii cząstek. \todo[inline]{reference} W symulacjach typu
    Particle-in-cell konieczne jest zastosowanie innych algorytmów. Dobrym
    algorytmem symplektycznym jest na przykład powszechnie znany
    \emph{leapfrog}, polegający na przesunięciu prędkości o połowę iteracji
    czasowej względem położeń.\todo[inline]{reference} Mimo tego, że energie
    cząstek w ruchu obliczonym tym integratorem nie są lokalnie stałe na
    krótkich skalach czasowych, to jednak zachowują energię na skali globalnej.

    \todo[inline]{IMAGE: chyba miałem to na coldplasma}

    W przypadku ruchu w polu magnetycznym nie wystarczy, niestety, użyć
    zwykłego algorytmu \emph{leapfrog}. \todo[inline]{READ} Używa się tutaj
    specjalnej adaptacji tego algorytmu na potrzeby ruchu w zmiennym polu
    elektromagnetycznym, tak zwanego integratora Borysa,
    \todo[inline]{REFERENCE} który rozbija pole elektryczne na dwa impulsy,
    między którymi następują dwie \todo[inline]{CHECK} rotacje polem
    magnetycznym. Algorytm jest dzięki temu symplektyczny i długofalowo
    zachowuje energię cząstek.

    \begin{equation}
        bo = ris
        \label{eqn:boris-pusher}
    \end{equation}
 \todo[inline]{boris pusher equation}
    W naszym przypadku dochodzi jeszcze jedno utrudnienie związane z
    relatywistycznością symulacji. \todo[inline]{stylistyka} Przed obliczeniem
    korekty prędkości konieczne jest przetransformowanie prędkości z układu
    ``laboratoryjnego'' $\vec{v}$ na prędkość w układzie poruszającym się z
    cząstką $\vec{u}$, czego dokonuje się poprzez parę transformacji:

    \begin{align}
        \vec{u} = \vec{v} \gamma
        \label{eqn:gamma-transformation}
    \end{align}
 \todo[inline]{finish this eq}
    \subsubsection{Depozycja gęstości ładunku i prądu} \todo[inline]{pick up here}
    Depozycja ładunku odbywa się w prosty sposób, przy następujących założeniach:
    \begin{itemize}
        \item Każda makrocząstka ma własny (wspólny wewnątrz \code{Species})
            ładunek $q$ oraz parametr \code{scaling} (również)
            \todo[inline]{STYLE?}
            decydujący o tym, ile rzeczywistych cząstek reprezentuje.
            Sumaryczny ładunek makrocząstki wynosi więc \code{q*scaling}
        \item Każda makrocząstka ma szerokość jednej komórki siatki $\Delta x$.
            Cząstka zlokalizowana więc środkiem
            w połowie długości komórki będzie w niej całkowicie zawarata.
        \item W ten sposób możemy stwierdzić, \todo[inline]{CONTINUE}
    \end{itemize}

    Powszechnie stosowana od zarania dziejów metod particle-in-cell
    \todo[inline]{refka: Dawson} jest interpolacja liniowa, polegająca na
    zdepozytowaniu w $i$-tej komórce siatki \todo[inline]{dokończyć}

    $1 = \sum_i S_i$ \todo[inline]{dokończyć wzór}

    W naszym przypadku wymagamy również, żeby depozycja prądu była spójna z
    depozycją ładunku, to znaczy zachowywała ładunek.

    \subsubsection{Interpolacja pól elektrycznego i magnetycznego}
    Interpolacja pól elektrycznego i magnetycznego odbywa się na bardzo podobnej zasadzie, co depozycja.
    Wartosci pol sa liniowo skalowane do pozycji czastek wedlug ich wzglednych położeń wewnątrz komórek.

    \todo[inline]{CONTINUE opowiadanie o interpolacji}
    \subsubsection{Field solver} \todo[inline]{przerobić}

    Ewolucja pola elektromagnetycznego opisana jest poprzez równania Maxwella.
    Jak pokazują Buneman i Villasenor, numerycznie można zastosować dwa główne
    podejścia: \todo[inline]{zredagować} 1. wykorzystać równania na dywergencję
    pola (prawa Gaussa) do rozwiązania pola na całej siatce. Niestety, jest to
    algorytm inherentnie globalny, w którym informacja o warunkach brzegowych
    jest konieczna w każdym oczku siatki
    \todo[inline]{alternatywa na słowo "oczko"?}
    2. wykorzystać równania na rotację pola (prawa Ampera i Faradaya), opisujące ewolucję czasową pól. Jak łatwo pokazać (Buneman),
    dywergencja pola elektrycznego oraz magnetycznego nie zmienia się w czasie pod wpływem tak opisanej ewolucji czasowej:

    Co za tym idzie, jeżeli rozpoczniemy symulację od znalezienia pola na
    podstawie warunków brzegowych i początkowych (gęstości ładunku), możemy już
    dalej iterować pole na podstawie równań rotacji. Ma to dwie znaczące
    zalety:
    
    * algorytm ewolucji pola staje się trywialny obliczeniowo,
    zwłaszcza w 1D - ogranicza się bowiem do elementarnych operacji lokalnego
    dodawania i mnożenia.
    
    * algorytm ewolucji pola staje się lokalny (do
    znalezienia wartości pola w danym oczku w kolejnej iteracji wykorzystujemy
    jedynie informacje zawarte w tym właśnie oczku i potencjalnie jego
    sąsiadach \todo[inline]{jak to faktycznie wygląda z tym algo?} co zapobiega
    problemowi informacji przebiegającej w symulacji szybciej niż światło oraz
    zapewnia stabilność na podstawie
    warunku Couranta.

    \todo[inline]{gładsze przejście tutaj - wyprowadzenie field solvera}
    W 1D można dokonać dekompozycji składowych poprzecznych pola
    elektromagnetycznego (tutaj oznaczanych $y$, $z$) na propagujące się w
    przód ($+$) i w tył ($-$) obszaru symulacji. Składowe $E_y$, $B_z$ są
    zebrane poprzez zamianę zmiennych w dwie wielkości elektrodynamiczne $F^+$,
    $F^-$.

    Wychodzimy z rotacyjnych równań Maxwella:

    \begin{equation}
        \nabla \times \vec{E} = -\frac{\partial \vec{B}}{\partial t}
        \nabla \times \vec{B} = \mu_0 (\vec{j} + \epsilon_0 \frac{\partial \vec{E}}{\partial t})
        \label{eqn:Maxwell-rotation-derivation}
    \end{equation}

    \todo[inline]{skończyć wyprowadzenie}

    \begin{equation}
        F^{+} = E_y + c B_z
        F^{-} = E_y - c B_z
        \label{eqn:Birdsall-electromagnetic-quantities}
    \end{equation}
    Analogicznie, dla składowych $E_z$, $B_y$:

    \todo[inline]{zweryfikować znaki i czy c nie jest w mianowniku}
    \begin{equation}
        G^{+} = E_z - c B_y
        G^{-} = E_z + c B_y
        \label{eqn:Birdsall-electromagnetic-quantities-alternate-axes}
    \end{equation}
    Wyrazem ``źródłowym'' dla F, G jest prąd poprzeczny. Po dyskretyzacji
    równania, wyrażenie na ewolucję pól F, G między iteracjami przybiera
    postać:

    \begin{equation}
        {F^{+}}^{n+1}_{i+1} = F^{+}_{n} + j
    \end{equation}
\todo[inline]{sprawdzić}
    Z tego powodu bardzo istotnym dla dokładności i stabilności algorytmu staje
    się sposób depozycji ładunku - należy pilnować, aby był robiony w sposób
    który spełnia zachowanie ładunku. Inaczej koniecznym staje się aplikowanie
    tak zwanej poprawki Borysa, \todo[inline]{źródło - prezentacja}
    aby upewnić się, że warunek z równań Maxwella $\nabla \rho / \varepsilon_0
    = \nabla \cdot \vec{E}$ jest wciąż spełniony.

    Składowa podłużna pola jest obliczana poprzez wyrażenie

    \begin{equation}
    \frac{\partial E_x}{\partial t} = - \frac{j_x}{\varepsilon_0}
    \label{longitudinal-field-differential}
    \end{equation}

    czy raczej jej dyskretny odpowiednik

    \begin{equation}
        E_i^{n+1} = E_i^n - \frac{\Delta t}{\varepsilon_0} j_{x,i}^{n+1/2}
    \label{longitudinal-field-finite-differential}
    \end{equation}


    \subsection{Warunki początkowe dla cząstek}

    W celu dobrania warunków początkowych wykorzystuje się algorytm opisany w
    .\todo[inline]{Birdsall Langdon} Jego działanie można łatwo zilustrować na
    przykładzie początkowej funkcji gęstości cząstek zadanej dowolną funkcją
    analityczną. \todo[inline]{czy analityczna nie jest słowem zarezerwowanych
    dla tych na szeregi}

    Używając funkcji dystrybucji w jednym wymiarze zależnej jedynie od
    położenia znormalizowanej do liczby cząstek $N$, można wykonać całkowanie
    kumulatywne po siatce gęstszej niż liczba cząstek na wybranym przedziale,
    po czym umieścić cząstki w miejscach, gdzie obliczona dystrybuanta funkcji
    przybiera kolejne większe całkowite wartości.

    \todo[inline]{rysunek: przykład z ipynb}

    Zaimplementowany algorytm jest w stanie przyjąć dowolną funkcję analityczną
    \todo[inline]{czy nie przesadzam?} i zrenormalizować ją tak, aby $\int_0^L
    f(x) dx = N$. W praktyce wykorzystuje się wartości marginalnie większe niż
    $N$, mianowicie $N+0.1$, co pozwala na uniknięcie problemów ze skończoną
    dokładnością obliczeń na liczbach zmiennoprzecinkowych.

    Aby uniknąć problemu w przypadku dwóch \code{Species} cząstek o identycznej
    liczbie makrocząstek i przeciwnym znaku które według powyższego algorytmu
    zostałyby rozłożone w identycznych miejscach z powodu niezależnego
    stosowania algorytmu dla każdej grupy cząstek, co prowadziłoby do
    neutralizacji ładunku na całej symulacji, na położenia cząstek nakłada się
    dodatkowy gaussowski szum o niewielkiej intensywności.

    Analogiczny algorytm znajduje zastosowanie w obliczaniu początkowych
    wartości prędkości dla cząstek.  Wykorzystuje się relatywistyczny rozkład
    Maxwella

    \begin{align}
        f(p) = \frac{N}{2 \pi} \frac{mc^2}{T} \frac{1}{1+T/mc^2} \exp \Big (\frac{-mc^2}{T}(\gamma -1) \Big)
        \gamma = \sqrt{1+p^2}
        \label{relativistic-maxwell-distribution}
    \end{align}

    Należy wspomnieć, że aby cząstki były prawidłowo ztermalizowane
    \todo[inline]{czy to jest słowo} należy zadbać o zdekorelowanie ich
    prędkości między sobą. Naiwne zastosowanie algorytmu na położenia prowadzi
    zaś do rozłożenia cząstek rosnąco numeracją w kierunku rosnącego położenia
    $x$.

    Rozwiązaniem tego problemu jest losowa zamiana prędkości między losowo
    wybranymi cząstkami.  \todo[inline]{dopisać jak będzie zrobione.}

    \subsection{Opis i treść kodu}
    Cały kod programu w celu reprodukowalności wyników tworzony był i jest
    dostępny na platformie Github \todo[inline]{link}

    \subsection{Wykorzystane biblioteki i technologie}

    \subsubsection{Numpy}
    \code{numpy} to biblioteka umożliwiająca wykonywanie złożonych obliczeń na
    n-wymiarowych macierzach bądź tablicach, utworzona w celu umożlwiienia
    zastąpienia operacjami wektorowymi iteracji po tablicach, powszechnie
    stosowanych w metodach numerycznych i będących znanym słabym punktem
    Pythona.
    \todo[inline]{REFERENCE źródło na powolność pętli}

    Pod zewnętrzną powłoką zawiera odwołania do znanych, wypróbowanych i
    sprawdzonych w numeryce modułów \code{LAPACK}, \code{BLAS} napisanych w
    szybkich, niskopoziomowych językach C oraz \code{FORTRAN}.  Jest to
    \emph{de facto} standard większości obliczeń numerycznych w Pythonie.

    Należy zauważyć, że operacje matematyczne w \code{Numpy} są automatycznie
    zrównoleglane \todo[inline]{refka intel MKL} tam, gdzie pozwala na to
    niezależność obliczeń.

    Numpy jest oprogramowaniem otwartym, udostępnianym na licencji BSD.
    \todo[inline]{refka}

    \subsubsection{scipy}
    Kolejną podstawową biblioteką w numerycznym Pythonie jest \code{scipy},
    biblioteka zawierająca wydajne implementacje wielu podstawowych algorytmów
    numerycznych służących między innymi całkowaniu, optymalizacji, algebrze
    liniowej czy transformatom Fouriera.  W naszym przypadku stosujemy zawarte
    w tej bibliotece funkcje całkujące do określenia początkowego profilu
    gęstości plazmy.  \todo[inline]{czy stosuję scipy gdzieś jeszcze}

    \subsubsection{Numba}
    \code{numba} to biblioteka służąca do kompilacji just-in-time kodu.
    \todo[inline]{Przerobić wyjaśnienie działania Numba} W wielu przypadkach
    pozwala na osiągnięcie kodem napisanym w czystym Pythonie wydajności
    marginalnie niższej bądź nawet równej do analogicznego programu w C bądź
    Fortranie. \todo[inline]{refka} Jednocześnie należy zaznaczyć prostotę jej
    użycia:

    \todo[inline]{fragment kodu. @jit przed kodem}


    \subsubsection{HDF5}
    HDF5 jest wysokowydajnym formatem plikow służącym przechowywaniu danych
    liczbowych w drzewiastej, skompresowanej strukturze danych, razem z
    równoległym, wielowątkowym zapisem tych danych.  W Pythonie implementuje go
    biblioteka h5py. \todo[inline]{reference h5py} Używa się go na przykład w
    \todo[inline]{lista miejsc gdzie używają hdf5}
    \todo[inline]{https://github.com/PPPLDeepLearning/plasma-python}

    W bieżącej pracy wykorzystuje się go do przechowywania danych numerycznych
    dotyczących przebiegu symulacji, pozwalających na ich dalsze przetwarzanie
    i analizę poprzez wizualizację.

    \subsubsection{matplotlib}
    Do wizualizacji danych z symulacji (oraz tworzenia schematów w sekcji
    teoretycznej niniejszej pracy) użyto własnoręcznie napisanych skryptów w
    uniwersalnej bibliotece graficznej \code{matplotlib}. \code{matplotlib}
    zapewnie wsparcia zarówno dla grafik statycznych w różnych układach
    współrzędnych (w tym 3D), jak również dla dynamicznie generowanych animacji
    przedstawiających przebiegi czasowe symulacji.

    Matplotlib również jest oprogramowaniem otwartym, udostępnianym na licencji
    \todo[inline]{matplotlib license, reference}

    \subsubsection{py.test}
    Przy pracy nad kodem użyto frameworku testowego \code{py.test}
    \todo[inline]{refenrece} Obsługa testów jest trywialna:

    \todo[inline]{przykład testu z programu}

    Należy zaznaczyć, że w numeryce, gdzie błędne działanie programu nie
    objawia się zazwyczaj błędem wykonywania programu, a jedynie błędnymi
    wynikami, dobrze zautomatyzowane testy jednostkowe potrafią zaoszczędzić
    bardzo dużo czasu na debugowaniu poprzez automatyzację uruchamiania
    kolejnych partii kodu i lokalizację błędnie działających części algorytmu.
    Dobrze napisane testy są praktycznie koniecznością w dzisiejszych czasach,
    zaś każdy nowo powstały projekt numeryczno-symulacyjny powinien je
    wykorzystywać, najlepiej do weryfikacji każdej części algorytmu z osobna.

    Dobrym przykładem skutecznego testu jednostkowego jest porównanie wyników z
    fragmentu algorytmu (na przykład depozycji ładunku, który to test zawarty
    jest w pliku) \code{pythonpic/tests/test\_current\_deposition.py}
    \todo[inline]{sprawdzić urla} z wynikami z poprzedniego, zweryfikowanego
    programu, bądź z obliczeniem ręcznym.

    \code{py.test} jest oprogramowaniem otwartym, dostępnym na licencji
    \todo[inline]{sprawdzić licencję}

    \subsubsection{Travis CI}
    Nieocenionym narzędziem w pracy nad kodem był system ciągłej integracji
    (\emph{continuous integration}) Travis CI \todo[inline]{refka} dostępny za
    darmo dla projektów open-source. Travis pobiera aktualne wersje kodu przy
    każdej aktualizacji wersji dostępnej na serwerze GitHub i uruchamia testy,
    zwracając komunikat o ewentualnym niepowodzeniu i pozwalając na jednoczesne
    uruchamianie bieżących, intensywnych symulacji przy jednoczesnym
    uruchamianiu lżejszych, acz wciąż zasobożernych \todo[inline]{to słowo}
    symulacji testowych i testów algorytmicznych.

    \subsection{snakeviz} W optymalizacji przydatny okazał się program
    \code{snakeviz} dostępny na licencji opensource i pozwalający na
    wizualizację wyników z profilowania symulacji. Pozwala w wygodny sposób
    zbadać, które fragmenty kodu najbardziej spowalniają symulację, które są
    najlepszymi kandydatami do optymalizacji, oraz jak skuteczne (bądź
    nieskuteczne) okazują się próby polepszenia ich wydajności.
    \todo[inline]{refka} \todo[inline]{grafika snakeviz}

    \subsection{Struktura i hierarchia kodu}

    Program ma obiektową strukturę zewnętrzną, którą w celu łatwości
    zrozumienia jego działania nakrywa wewnętrzną warstwę składającą się
    głównie z n-wymiarowych tablic \code{numpy.ndarray} oraz zwektoryzowanych
    operacji na nich.

    Część symulacyjna kodu składa się z kilku prostych koncepcyjnie elementów:

    \subsubsection{Grid}
    Klasa reprezentująca dyskretną siatkę Eulera, na której dokonywane są
    obliczenia dotyczące pól elektromagnetycznych oraz gęstości ładunku i
    prądu.  Zawiera:
    \begin{itemize}
        \item $x_i$ - tablicę położeń lewych krawędzi komórek siatki
        \item $N_G$ - liczbę komórek siatki
        \item $T$ - sumaryczny czas trwania symulacji
        \item $\Delta x$ - krok przestrzenny siatki - $N_G * \Delta x$ daje
            długość obszaru symulacji
        \item $\rho_i$ - tablicę gęstości ładunku na siatce.
        \item $\vec{j}_{i,j}$ - tablicę gęstości prądu na siatce.
        \item $E_{i,j}$ - tablicę pola elektrycznego na siatce.
        \item $B_{i,j}$ - tablicę pola magnetycznego na siatce.
        \item $c$, $\varepsilon_0$ - stałe fizyczne - prędkość światła oraz
            przenikalność elektryczną próżni.
        \item $\Delta t$ - krok czasowy symulacji, obliczony jako $\Delta t =
            \Delta x / c$.
        \item $N_T$ - liczbę iteracji czasowych symulacji.
        \item \code{BC} - \emph{Boundary Condition}, funkcję czasu określającą
            wartość warunku brzegowego dotyczącego natężenia fali
            elektromagnetycznej (laserowej) wchodzącej do pola symulacji z
            lewej strony.
    \end{itemize}

    Istotne metody klasy \code{Grid}, o których należy wspomnieć, to:
    \begin{itemize}
         \item \code{apply\_bc} - aktualizuje krańcowe wartości tablic $E$, $B$
             w oparciu o podany warunek brzegowy.
         \item gather\_current \todo[inline]{finish these}
         \item gather\_charge
         \item solve
         \item field\_solve
         \item electric\_field\_function, magnetic
         \item save\_to\_h5py
    \end{itemize}

    \subsubsection{Species}
    Klasa reprezentująca pewną grupę makrocząstek o wspólnych cechach, takich
    jak ładunek bądź masa.  Przykładowo, w symulacji oddziaływania lasera z
    tarczą wodorową jedną grupą są protony, zaś drugą - elektrony.  Do
    zainicjalizowania wymaga instancji \code{Grid}, z której pobiera informacje
    takie jak stałe fizyczne $c$, $\varepsilon_0$, liczbę iteracji czasowych
    $N_T$ i czas trwania iteracji $\Delta t$.

    Zawiera skalary:
    \begin{itemize}
        \item $N$ - liczba makrocząstek
        \item $q$ - ładunek cząstki
        \item $m$ - masa cząstki
        \item \code{scaling} - liczba rzeczywistych cząstek, jakie reprezentuje
            sobą makrocząstka. Jej sumaryczny ładunek wynosi $q *
            $\code{scaling}, masa $m * $\code{scaling}.
        \item \code{N\_alive} - liczba cząstek obecnie aktywnych w symulacji.
            Zmniejsza się w miarę usuwania cząstek przez warunki brzegowe.
    \end{itemize}

    Poza skalarami zawiera tablice rozmiaru $N$:
    \begin{itemize}
        \item jednowymiarowych położeń makrocząstek $x^n$, zapisywanych w
            iteracjach $n, n+1, n+2$\ldots
        \item trójwymiarowych prędkości makrocząstek $\vec{v}^{n+\frac{1}{2}}$,
            zapisywanych w iteracjach $n+\frac{1}{2}, n+{3}{2}, n+{5}{2}$\ldots
        \item stanu makrocząstek (flagi boolowskie oznaczające cząstki aktywne
            bądź usunięte z obszaru symulacji)
    \end{itemize}

    Poza tym, zawiera też informacje dotyczące zbierania danych diagnostycznych
    dla cząstek, niepotrzebnych bezpośrednio w czasie symulacji:
    \begin{itemize}
        \item \code{name} - słowny identyfikator grupy cząstek, dla potrzeb legend wykresów
        \item $N_T$ - liczbę iteracji czasowych w symulacji
        \item $N_T^s$ - zmniejszoną liczbę iteracji, w których następuje pełne
            zapisanie położeń i prędkości cząstek.  Dane te są wykorzystywane
            do tworzenia diagramów fazowych cząstek.
        \item odpowiadające poprzednio wymienionym tablice rozmiaru $(N_T^s,
            N)$, $(N_T^s, N, 3)$.  \item jedną tablicę rozmiaru $(N_T, N_G)$
            dotyczącą zebranym podczas
            depozycji ładunku informacjom diagnostycznym o przestrzennej
            gęstości cząstek.
        \item trzy tablice rozmiaru $(N_T)$ dotyczącą średnich prędkości,
            średnich kwadratów prędkości i odchyleń standardowych prędkości.
    \end{itemize}

    Jeżeli liczba makrocząstek lub iteracji przekracza pewną stałą, dane zapisywane są jedynie dla co $n$-tej cząstki,
    gdzie $n$ jest najniższą liczbą całkowitą która pozwala na zmniejszenie tablic poniżej tej stałej.

    Warto wspomnieć o metodach klasy \code{Species}:
    \begin{itemize}
        \item push \todo[inline]{fill these}
    \end{itemize}

    \subsubsection{Simulation}
    Klasa zbierająca w całość Grid oraz dowolną liczbę Species zawartych w
    symulacji, jak również pozwalająca w prosty sposób na wykonywanie iteracji
    algorytmu i analizy danych. Jest tworzona tak przy uruchamianiu symulacji,
    jak i przy wczytywaniu danych z plików \code{.hdf5}.

    \begin{itemize}
        \item $\Delta t$ - krok czasowy
        \item $N_T$ - liczba iteracji w symulacji
        \item \code{Grid} - obiekt siatki
        \item \code{list\_species} - lista grup makrocząstek w symulacji
    \end{itemize}
    \todo[inline]{metody simulation}

    Przygotowanie warunków początkowych do danej symulacji polega na utworzeniu
    nowej klasy dziedziczącej po \code{Simulation}, która przygotowuje siatkę,
    cząstki i warunki brzegowe zgodnie z założeniami eksperymentu i wywołuje
    konstruktor \code{Simulation}.  Należy również przeciążyć metodę
    \code{grid\_species\_init}, która przygotowuje warunki początkowe. Domyślna
    wersja tej metody wykonuje pierwszą, początkową iterację równań ruchu,
    która pozwala na zachowanie symplektyczności integratora równań ruchu,
    \todo[inline]{stylistyka?} co pomaga zachować energię cząstek w symulacji.

    Aby uruchomić symulację, należy wywołać jedną z metod:
    \begin{itemize}
         \item \code{run} - podstawowy cykl obliczeń, używany do pomiarów
             wydajności programu
         \item \code{test\_run} - obliczenia oraz obróbka danych na potrzeby
             analizy, głównie stosowana w testach
         \item \code{lazy\_run} - \code{test\_run} z zapisem do pliku oraz
             wczytaniem z pliku \emph{.hdf5}, jeżeli początkowe warunki oraz
             wersja kodu zgadzają się. W przeciwnym razie symulacja zostaje
             uruchomiona na nowo.
    \end{itemize}

    \subsubsection{Pliki pomocnicze}
    Poza powyższymi program jest podzielony na pliki: \todo[inline]{aktualizacja}
    \begin{itemize}
        \item algorithms\_grid - zawiera algorytmy dotyczące rozwiązywania
            równań Maxwella na dyskretnej siatce
        \item algorithms\_interpolation - zawiera algorytmy interpolujące pola
            z cząstek na siatkę i odwrotnie
        \item algorithms\_pusher - zawiera algorytmy integrujące numerycznie
            równania ruchu cząstek
        \item animation - tworzy animacje dla celów analizy danych
        \item static\_plots - tworzy statyczne wykresy dla celów analizy danych
        \item plotting - zawiera ustawienia dotyczące analizy danych
            \todo[inline]{czy to można przenieść do simulation czy gdzieś?}
    \end{itemize}

    Przygotowane konfiguracje istniejących symulacji są zawarte w plikach
    \code{configs/run\_*}: \todo[inline]{przeformułować}
    \begin{itemize}
        \item run\_coldplasma
        \item run\_twostream
        \item run\_wave
        \item run\_beam
        \item run\_laser
    \end{itemize}

    Algorytmiczne testy jednostkowe są zawarte w katalogu \code{tests}.
