\section{Część analityczno-teoretyczna} % 30% pracy - opis problematyki podjętego tematu w zakresie wykorzystanym w pracy i analizie

\subsection{Fizyka plazmy}

Plazma, powszechnie nazywana czwartym stanem materii, to zbiór zjonizowanych %TODO: formalna definicja plazmy
cząstek oraz elektronów. Plazmy występują w całym wszechświecie, od materii międzygwiezdnej po błyskawice.
Ich istnienie uwarunkowane jest obecnością wysokich energii, wystarczających do zjonizowania atomów gazu.

Fizyka plazmy jest stosunkowo młodą nauką, której rozwój nastąpił dopiero w ostatnim stuleciu, zaczynając od badań Alfvena. %TODO reference
Globalny wzrost zainteresowania fizyką plazmy rozpoczął się w latach '50 ubiegłego wieku, %TODO - zweryfikować
gdy uświadomiono sobie, że można zastosować ją do przeprowadzania kontrolowanych reakcji syntezy jądrowej, % TODO: reference: fusion in europe history of fusion
które mogą mieć zastosowania w energetyce jako następny etap rozwoju po reakcjach rozpadu wykorzystywanych
w "klasycznych" elektrowniach jądrowych.

Poza tym plazmy mają szerokie zastosowanie w przemyśle metalurgicznym, elektronicznym, kosmicznym itp. %TODO: to potrzebuje źródła

\subsection{Modelowanie i symulacja plazmy}

Zjawiska z zakresu fizyki plazmy są jednymi z bardziej złożonych problemów modelowanie komputerowej.
Głównym, koncepcyjnie, powodem uniemożliwiającym zastosowanie prostych metod symulacji
znanych z newtonowskiej dynamiki molekularnej jest mnogość oddziaływań - każda cząstka oddziałowuje
z każdą inną nawzajem poprzez niepomijalne na dużych odległościach oddziaływania kulombowskie $\approx r^{-2}$.

Z powodu dużej liczby cząstek w układach plazmowych, jedynymi praktycznymi podejściami są opisy statystyczne, opierające się na
modelach kinetycznych. Wielkością opisującą plazmę jest tu funkcja dystrybucji

\begin{equation}
f(\vec{x}, \vec{v}, t)
\end{equation}

opisująca gęstość rozkładu plazmy w sześciowymiarowej przestrzeni fazowej (po trzy wymiary na położenia oraz prędkości).

Podstawowym równaniem statystycznym opisującym plazmę jest równanie Vlasova % TODO: może zacząć od Klimontowicza, jest wyprowadzalne z niego

\begin{equation}
f_{\alpha}
% TODO: wzór na równanie Vlasova
\end{equation}

W praktyce jest ono również nierozwiązywalne. Jednym z powodów jest koniecznośc uzyskania dobrej rozdzielczości prędkości
przy jednoczesnym zachowaniu zakresów obejmujących prędkości relatywistyczne. % TODO: runaway electrons

W modelowaniu komputerowym plazmy stosuje się dwa główne podejścia:
\begin{enumerate}
\item modele płynowe oparte na ciągłym opisie plazmy poprzez uśrednienie po dystrybucji
wielkości termodynamicznych, co daje modele takie jak magnetohydrodynamikę %TODO: reformulate
\item modele dyskretne oparte na samplowaniu dystrybucji plazmy przy użyciu dyskretnych cząstek
\end{enumerate}

Prawdopodobnie najpopoularniejszym modelem z tej drugeij kategorii są modele Particle-in-cell.

\subsection{Modele Particle-in-cell}

Idea modelu particle-in-cell jest wyjątkowo prosta i opiera się na idei przyspieszenia najbardziej złożonego obliczeniowo kroku
symulacji dynamiki molekularnej, czyli obliczania sił międzycząsteczkowych. Cząstki poruszają się w ciągłej, Lagrange'owskiej przestrzeni.
Ich ruch wykorzystywany jest do zebrania informacji dotyczącej gęstości ładunku i prądu na dyskretną, Eulerowską siatkę. Na siatce rozwiązane
są (jako równania różniczkowe cząstkowe) równania Maxwella, dzięki którym otrzymuje się pola elektryczne i magnetyczne, które z powrotem są przekazane
do położeń cząstek. Obliczeniowo, uwzględniając koszty odpowiednich interpolacji, pozwala to zredukować złożoność kroku obliczenia sił międzycząsteczkowych
do $n \log{n}$ z $n^2$ % TODO: wyrazić złożoność PIC przez rozmiar siatki

Algorytm particle-in-cell składa się z czterech elementów % GRAFIKA: cykliczny schemacik
\begin{enumerate}
\item GATHER \\
depozycja ładunku oraz prądu z położeń cząstek do lokacji na dyskretnej siatce poprzez interpolację,
co pozwala na sprawne rozwiązanie na tej siatce
równań Maxwella jako układu różnicowych równań cząstkowych zamiast obliczania skalujących się kwadratowo w liczbie cząstek
oddziaływań kulombowskich między nimi.
\item SOLVE \\
Sprawne rozwiązanie Maxwella na dyskretnej, Eulerowskiej siatce; znalezienie pól elektrycznego i magnetycznego
na podstawie gęstości ładunku i prądu na siatce.
\item SCATTER \\
Interpolacja pól z siatki do lokacji cząstek, co pozwala określić siły elektromagnetyczne działające na cząstki.
\item PUSH \\
iteracja równań ruchu cząstek na podstawie ich prędkości (aktualizacja położeń)
oraz działających na nie sił elektromagnetycznych (aktualizacja prędkości).
\end{enumerate}

\subsubsection{Makrocząstki}
Należy zauważyć, że obecnie nie jest możliwe dokładne odwzorowanie dynamiki układów plazmowych w sensie interakcji
między poszczególnymi cząstkami ze względu na liczbę cząstek % rzędu liczby Avogadro ~10^23
W tym kontekście bardzo szczęśliwym jest fakt, że wszystkie istotne wielkości zależą nie od ładunku ani masy,
ale od stosunku $q/m$. W praktyce stosuje się więc \emph{makrocząstki}, obdarzone ładunkiem i masą będące wielokrotnościami
tych wielkości dla cząstek występujących w naturze (jak jony i elektrony, pozwalając jednocześnie zachować gęstości
cząstek i ładunku % oraz inne wielkości fizyczne)
zbliżone do rzeczywistych.

\subsection{Problem testowy}

Problemem testowym, jakiego używamy do przetestowania wydajności działania algorytmu jest
interakcja impulsu laserowego z tarczą składającą się ze zjonizowanego wodoru i elektronów.

Układ ten modelowany jest jako jednowymiarowy.

Jest to tak zwany w literaturze model 1D-3D. O ile położenia cząstek są jednowymiarowe


\subsection{Python}
Python jest wysokopoziomowym, interpretowanym językiem programowania, którego atutami są szybkie prototypowanie,

Python znajduje zastosowania w analizie danych, uczeniu maszynowym (zwłaszcza w astronomii). W zakresie symulacji
w ostatnich czasach powstały kody skalujące się nawet w zakres superkomputerów, np. w mechanice płynów % TODO: PyFR

Atutem Pythona w wysokowydajnych obliczeniach jest łatwość wywoływania w nim zewnętrznych bibliotek napisanych
na przykład w C lub Fortranie, co pozwala na osiągnięcie podobnych rezultatów wydajnościowych jak dla kodów
napisanych w C.

\subsection{Optymalizacja kodu}
